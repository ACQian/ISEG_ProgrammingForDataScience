\chapter{Data Visualization}

\section{Matplotlib}

Matplotlib is a powerful and versatile Python library for creating static.
It supports different chart types including scatter plots, line charts, and bar charts.
While highly flexible, it can sometimes feel verbose for simple visualizations.
It serves as the backbone for other libraries like Seaborn and Plotly, which build on its capabilities to provide more user-friendly interfaces for specific types of visualizations.

Strengths:
\begin{itemize}
    \item Control over plot elements
    \item Wide range of plot types
    \item Suitable for publication-quality figures
\end{itemize}

Weaknesses:
\begin{itemize}
    \item Can be verbose for simple plots
    \item Steeper learning curve for beginners
\end{itemize}

The building blocks of Matplotlib include:
\begin{itemize}
    \item \textbf{Figure}: The overall container for the plot. Includes one or more subplots. It is like a blank canvas where multiple charts can be arranged.
    \item \textbf{Axes}: The area where the data is plotted, which can contain multiple subplots each displaying a different visualization.
    \item \textbf{Subplot}: A specific area within the figure where a plot is drawn, allowing for multiple plots in one figure. Useful for comparing different datasets side by side.
    \item \textbf{Axis}: Corresponds to the x and y axes of a plot, which can be customized with labels, ticks, and limits to enhance readability and presentation.
\end{itemize}

\begin{lstlisting}
    import matplotlib.pyplot as plt

    # Figure

    fig = plt.figure(figsize=(8, 6)) # creates a new figure with a specified size of 8 inches by 6 inches. This figure serves as the canvas for plotting.
    ax = fig.add_subplot(111) # adds a single subplot to the figure. The '111' argument indicates that the subplot should occupy the entire figure (1 row, 1 column, 1st subplot).
    plt.show() # displays the figure with the subplot. Since no data has been plotted yet, it will show an empty plot.

    # Axes
    fig, ax = plt.subplots() # creates a new figure and a single set of axes (subplot) in one step. This is a more concise way to create a figure and axes compared to the previous method.
    ax.plot([1, 2, 3], [4, 5, 6]) # plots a line graph on the axes using the provided x and y data points. The x values are [1, 2, 3] and the corresponding y values are [4, 5, 6].
    plt.show() # displays the figure with the plotted line graph.

    # Subplots
    fig, (ax1, ax2) = plt.subplots(1, 2) # creates a new figure with two subplots arranged in one row and two columns. The variables ax1 and ax2 refer to the first and second subplot, respectively.
    ax1.plot([1, 2, 3], [4, 5, 6], label="Dataset 1") # plots a line graph on the first subplot (ax1) using the provided x and y data points.
    ax2.scatter([1, 2, 3], [4, 5, 6], color="r", label="Dataset 2") # plots a scatter graph on the second subplot (ax2) using the same x and y data points.
    plt.show() # displays the figure with both subplots: the first showing a line graph and the second showing a scatter graph.

    # Axis
    fig, ax = plt.subplots() # creates a new figure and a single set of axes (subplot) in one step.
    ax.plot([1, 2, 3], [4, 5, 6]) # plots a line graph on the axes using the provided x and y data points.
    ax.set_xlabel("X-axis Label") # sets the label for the x-axis to "X-axis Label", which helps to identify what the x-axis represents.
    ax.set_ylabel("Y-axis Label") # sets the label for the y-axis to "Y-axis Label", which helps to identify what the y-axis represents.
    ax.set_title("Line Graph") # sets the title of the plot to "Line Graph", providing a descriptive heading for the visualization.
    plt.show() # displays the figure with the plotted line graph, including the x and y axis labels and the title.
\end{lstlisting}

\section{Other libraries}

Can also use the combination of Pandas and Seaborn for quick and easy data visualization, especially for statistical plots.