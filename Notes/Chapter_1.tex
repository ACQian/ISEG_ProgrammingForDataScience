\chapter{Class and Methods}

Defining a function within a class is called a method.
We can also create an object and call a method.
\begin{lstlisting}[language=Python]
    class Person:
        def speak(self):
            print("Hello!")

    p = Person()
    p.speak()  # Output: Hello!
\end{lstlisting}


The init method is a special method that is called when an object is instantiated.
\begin{lstlisting}[language=Python]
    class Person:
        def __init__(self, name):
            self.name = name

        def speak(self):
            print("Hello, my name is", self.name)

    p = Person("Alice")
    p.speak()  # Output: Hello, my name is, Alice
\end{lstlisting}


Inheritance allows a class to inherit attributes and methods from another class.

Privacy of the class can be controlled by prefixing an attribute or method name with an underscore (\texttt{\textunderscore}).
This indicates that the attribute or method is intended for internal use only.

\begin{lstlisting}[language=Python]
    class Person:
        def __init__(self, name):
            self._name = name  # Private attribute 
        def _speak(self):  # Private method
            print("Hello, my name is", self._name)
    p = Person("Alice")
    p._speak()  # Output: Hello, my name is, Alice
\end{lstlisting}