\chapter{Object Oriented Programming}

\section{Object oriented approach}

\dfn[]{Object Oriented Programming}{
    Object oriented programming (OOP) is a imperative programming programming paradigm
    where instructions are grouped together with the part of the state they operate on.
}

The main \textbf{structural components} of all systems are
\begin{itemize}
    \item Objects: Object is something that takes up space in the real or conceptual world with which somebody may do things.
    An object is an instance of a class.
    An object has three main characteristics:
    \begin{itemize}
        \item Name (or ID): Unique identifier for the object.
        \item State: Attributes that describe the object.
        \item Operations (or behavior): Actions that the object can perform.
    \end{itemize}
    \item Class objects: Class is the blueprint of an object.
\end{itemize}

Main \textbf{characteristics} of the approach are
\begin{itemize}
    \item Encapsulation: The mechanism of hiding the implementation of the object. Only the necessary details are exposed to the user.
    \item Abstraction: A principle which consists of ignoring the aspects of a subject that is not relevant for the present purpose.
    It is the process of simplifying complex systems by modeling classes based on the essential properties and behaviors an object must have.
    \item Inheritance: Mechanism by which one class can inherit attributes and methods from another class. Making new classes from existing ones.
    \item Polymorphism: Ability to present the same interface for different data types.
    Same function name but different signatures being used for different types.
\end{itemize}

\section{Class}

\dfn[]{Class Diagram}{
    A class diagram is a type of static structure diagram that describes the structure of a system by showing the system's classes, their attributes, operations (or methods), and the relationships among objects.
}
Elements of a \textbf{class}:
\begin{itemize}
    \item Class Name (ID Class): The name of the class. Nouns associated with the textual description of a problem. Singular.
    \item Attributes: The properties or characteristics of the class. Types include Real, Integer, Text, Boolean, Enumerated, etc.
    \item Operations (Methods): The functions or operations that can be performed on the class. Behaviors of the class.
\end{itemize}
\textbf{Relations} between classes:
\begin{itemize}
    \item Association: A relationship between two classes that indicates how objects of one class are connected to objects of another class.
    \item Aggregation: A special type of association that represents a "whole-part" relationship between classes.
    \item Composition: A stronger form of aggregation that implies ownership and a whole-part relationship where the part cannot exist independently of the whole.
    \item Generalization: A relationship between a more general class (superclass) and a more specific class (subclass) that indicates inheritance.
\end{itemize}
\begin{center}
    \includegraphics[scale=0.4]{Images/1.png}
\end{center}

\subsection{Class and Method in Python}

\begin{lstlisting}[
    caption={Empty class definition in Python}
]
    class Person:
        pass # An empty class definition

    p = Person()
    print(p)  # Output: <__main__.Person object at 0x...>
\end{lstlisting}

\begin{lstlisting}[
    caption={Method}
]
    """Define class with method"""
    class Person:
        def speak(self):
            print("Hello!")

    """Create object and call method"""
    p = Person()
    p.speak()  # Output: Hello!
\end{lstlisting}

\begin{lstlisting}[caption={Attributes and \_\_init\_\_ method}]
    """
    The __init__ method is a special method in Python classes.
    It is a method that Python calls when you create a new instance of this class.
    """
    class Person:
        def __init__(self, name):
            self.name = name  # Initialize the name attribute
        def speak(self):
            print('Hello, my name is', self.name)

    p = Person('Carlos')
    p.speak()  # Output: Hello, my name is Carlos
\end{lstlisting}

\nt{
    The first argument of every class method, including init, is always a reference to the current instance of the class.
    By convention, this argument is always named self.
}

\begin{lstlisting}[caption={Class Pet with attributes and methods}]
    class Pet(object):
        def __init__(self, name, species):
            self.name = name
            self.species = species
        def getName(self):
            return self.name
        def getSpecies(self):
            return self.species
        def __str__(self):
            return "%s is a %s" % (self.name, self.species)
\end{lstlisting}

\begin{lstlisting}[caption={Inheritance}]
    class Dog(Pet):

    """Dog inherits from Pet"""

        def __init__(self, name, chases_cats):
            # Call the parent class (Pet) constructor
            Pet.__init__(self, name, "Dog")
            self.chases_cats = chases_cats

        def chasesCats(self):
            return self.chases_cats

    class Cat(Pet):

    """Cat inherits from Pet"""

        def __init__(self, name, hates_dogs):
            # Call the parent class (Pet) constructor
            Pet.__init__(self, name, "Cat")
            self.hates_dogs = hates_dogs

        def hatesDogs(self):
            return self.hates_dogs
\end{lstlisting}

\begin{center}
    \includegraphics[scale=0.5]{Images/2.png}
    \includegraphics[scale=0.5]{Images/3.png}
\end{center}

\ex[]{Inheritance}{
    myPet = Pet(``Boby'', ``Dog'')\\
    myDog = Dog(``Rex'', True)
    \begin{itemize}
        \item isinstance(myDog, Pet)  \# Returns: True
        \item isinstance(myDog, Dog)  \# Returns: True
        \item isinstance(myPet, Pet)  \# Returns: True
        \item isinstance(myPet, Dog)  \# Returns: False
    \end{itemize}
}

\subsection{Access Modifiers}
There are three types of access modifiers in Python:
\begin{itemize}
    \item Public: Members (attributes and methods) declared as public are accessible from anywhere.
    \item Protected: Members declared as protected are accessible within the class and its subclasses. In Python, this is indicated by a single underscore prefix (e.g., \_attribute).
    \item Private: Members declared as private are accessible only within the class itself. In Python, this is indicated by a double underscore prefix (e.g., \_\_attribute).
\end{itemize}

\begin{lstlisting}[caption={Private Attributes}]
    class Person:
        def __init__(self, name, age):
            self.__name = name      # Private attribute
            self.__age = age        # Private attribute
    
    p = Person("David", 23)
    p.__name  # This will raise an AttributeError
\end{lstlisting}

\begin{lstlisting}[caption={Public Attributes}]
    class Person:
        def __init__(self, name, age):
            self.name = name      # Public attribute
            self.age = age        # Public attribute

    p = Person("David", 23)
    p.name  # This will work fine and return "David"
\end{lstlisting}

\begin{lstlisting}[caption={Protected Attributes}]
    class Person:
        def __init__(self, name, age):
            self._name = name      # Protected attribute
            self._age = age        # Protected attribute

    p = Person("David", 23)
    p._name  # This will work, but it's discouraged to access protected members directly
\end{lstlisting}

\begin{lstlisting}[caption={Example Usage}]
    class Person:
        def __init__(self, money=0, energy=100):
            self.money = money
            self.energy = energy
        
        def work(self, hours):
            if self.energy >= hours * 10:
                self.money += hours * 10 # Assume earning $10 per hour of work
                self.energy -= hours * 10
                print (f"Worked for {hours} hours. Money increased to ${self.money}, energy decreased to {self.energy}.")
            else:
                print("Not enough energy to work.")
\end{lstlisting}

%\dfn{Definition Topic}{Definition Statement}
%\thm{Theorem Name}{Theorem Statement}
%\cor[cori]{Corollary Name}{Corollary Statement}
%\lem{Lemma Name}{Lemma Statement}
%\clm{Claim Name}{Claim Statement}
%\ex{Example Name}{Example explained}
%\opn{Open Question Name}{Question Statement}
%\pr{Question Name}{Question Statement}
%\nt{Special Note}
%\wc{Wrong Concept topic}{Explanation}
%\proof{Proof Idea}{}
